\documentclass[12pt]{article}
\usepackage[margin=1.0in]{geometry}
\usepackage{graphicx}
\usepackage{ amssymb }
\usepackage{listings}
\usepackage[toc,page]{appendix}
\usepackage{wrapfig}
\usepackage{float}
\usepackage{indentfirst}
\usepackage{subcaption}
\usepackage{mwe}
\renewcommand{\thesubsection}{\thesection.\alph{subsection}}



\title{CTA200 2021 Assignment 2}
\author{Milica Ivetic \\ milica.ivetic@mail.utoronto.ca}
\date{}

\begin{document}

\maketitle


\section*{[1] Question 1}

For this question I wrote two functions, one for each of the approximation methods given in the question. I then plotted the error compared to the analytical derivative of $f(x) = sin(x)$ using the formula given in the question. The loglog plot for the two different approximation methods is given in Figure 1.  

\begin{figure}[h]
\begin{center}
\includegraphics[scale=0.7]{Question1plot.pdf}
\caption{The loglog plot of error compared to the analytical derivative for both approximation methods. \label{fig1}}
\end{center}
\end{figure}

In the above plot we see that the first approximation method (blue line) has a changing slope, suggesting that the bigger the value for h gets, the less accurate the approximation is. For the second approximation method (orange line), the slope remains constant, suggesting that this method retains its accuracy for increasing h values. Also, for both methods, error increases as h increases.


\section*{[2] Question 2}

Three functions were needed for this question. The first function iterated the equation $z_{i + 1} = z_i^2 + c$ up to a certain arbitrary value. The chosen number of iterations was 30. 

The second function plots an image in which the points in the complex plane that diverge are given a certain colour (blue) and bounded points are given another colour (orange), as seen in Figure 2a.


\begin{figure*}[h]
        \centering
        \begin{subfigure}[b]{0.49\textwidth}
            \centering
            \includegraphics[width=\textwidth]{divbound.pdf}
            \caption[]%
            {{\small Divergent and bounded points on the complex plane.}}    
            \label{fig:mean and std of net14}
        \end{subfigure}
        \begin{subfigure}[b]{0.5\textwidth}   
            \centering 
            \includegraphics[width=\textwidth]{colourmap.pdf}
            \caption[]%
            {{\small Divergence of z on complex plane. The numbers on the colourbar represent the number of iterations.}}    
        \end{subfigure}
        \caption[]%
        {\small Plots for Question 2.} 
    \end{figure*}


The third function produces an image where the points are coloured by a colour scale that indicates the iteration number at which the given point diverged. As shown in Figure 2b, the smaller the number is on the colour bar, the faster it diverges. This is because the number represents the number of iterations before it diverges, hence why the outer parts of the plot are darker purple.



\section*{[3] Question 3}

For this question, the Scipy ODE integrator was used. Figure 3 shows the plots of solutions with varying $\gamma$ and $\beta$ values. Also, the initial conditions used were $I(0) = 1, S(0)=999, R(0) = 0$.


\begin{figure*}[h]
        \centering
        \begin{subfigure}[b]{0.49\textwidth}
            \centering
            \includegraphics[width=\textwidth]{gamma = 0.1 and beta = 1.pdf}
            \caption[]%
            {{\small Solutions for $\gamma = 0.1$ and $\beta = 1$.}}    
            \label{fig:mean and std of net14}
        \end{subfigure}
        \begin{subfigure}[b]{0.5\textwidth}   
            \centering 
            \includegraphics[width=\textwidth]{gamma = 0.5 and beta = 0.5.pdf}
            \caption[]%
            {{\small Solutions for $\gamma = 0.5$ and $\beta = 0.5$.}}    
        \end{subfigure}
        \begin{subfigure}[b]{0.5\textwidth}   
            \centering 
            \includegraphics[width=\textwidth]{gamma = 1 and beta = 0.5.pdf}
            \caption[]%
            {{\small Solutions for $\gamma = 1$ and $\beta = 0.5$.}}    
        \end{subfigure}
        \caption[]%
        {\small Plots for Question 3.} 
    \end{figure*}




\end{document}
